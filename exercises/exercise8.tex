\documentclass[a4paper]{article}
\usepackage{amsmath}
\usepackage{url}
\usepackage{graphicx}
\usepackage[margin=1.2in]{geometry}

\title{\vspace{-3em} Exercise 8\\[10pt] \large FY8503 Advanced theoretical physics\\  Transport modelling with Stochastic differential equations}

\author{November 11, 2023}
\date{}

\begin{document}
\maketitle



\section*{Problem 1}

In this exercise, we will look at the Gillespie algorithm and the coagulation equation, both of which were presented and implemented in the previous lecture (notebook 10 in the github repo).

\subsection*{Task a}

The implementation of the coagulation equation presented in the lecture used a double for loop to check all particle pairs for coagulation events during a timestep. For such an approach, the simulation run time will scale as $N_p^2$, where $N_p$ is the number of particles, which is not very practical. The task is to try rather to write an implementation of a solver for this equation that makes use of Gillespies method.

It may be useful to recall that with a constant kernel function, all pairs of particles are equally likely to collide, which simplifies the implementation somewhat.


\section*{References}

Gillespie, D. T. (2007). ``Stochastic simulation of chemical kinetics''. \emph{Annual Review of Physical Chemistry}, 58, 35-55. \url{https://doi.org/10.1146/annurev.physchem.58.032806.104637}


\end{document}
