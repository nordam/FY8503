\documentclass[a4paper]{article}
\usepackage{amsmath}
\usepackage[margin=1.2in]{geometry}

\title{\vspace{-3em} Exercise 2\\[10pt] \large FY8503 Advanced theoretical physics\\  Transport modelling with Stochastic differential equations}

\author{September 18, 2023}
\date{}

\begin{document}
\maketitle

\section*{Problem 1}

Implement the Euler-Maruyama method, and use it to solve the simplest SDE from the last lecture:
\begin{align}
    \label{eq:}
    \mathrm{d} X_t = a \, \mathrm{d}t + b \, \mathrm{d}W_t,
\end{align}
where the drift and diffusion terms, $a$ and $b$, are constants.

As this is just the Wiener process scaled by a constant $b$ and shifted by an amount $at$, we can see that the distribution of a large number of realisations $X_t$, at time $t$, will follow a Gaussian distribution with some mean $\mu$ and variance $\sigma^2$.

\subsubsection*{Task a}
Solve for a large number of realisations, and plot a histogram of values of $X_t$ together with the expected Gaussian distribution, for a few different times, $t$.

\subsubsection*{Task b}
A more rigorous way (than visually comparing the expected distribution to a histogram) to look at the convergence to a distribution is perhaps to look at the moments of the distribution, for example the mean (first moment) or the variance (central second moment).

Solve the same equation for $N_p$ realisations and a timestep $\Delta t$, and compare the mean and variance of the distribution of solutions to the mean and variance of the expected Gaussian distribution. Try different values of $N_p$, and different values of $\Delta t$. Do you observe any trends?

\section*{Problem 2}

Use the Euler-Maruyama method to solve the other SDE we looked at in the last lecture:
\begin{align}
    \label{eq:}
    \mathrm{d} X_t = a X_t \, \mathrm{d}t + b X_t \, \mathrm{d}W_t,
\end{align}
where the drift and diffusion terms, $a X_t$ and $b X_t$, are now proportional to $X_t$. We recall from the lecture (see also Øksendal, 2003, p. 67) that the expectation value of the solution is 

\begin{align}
    \label{eq:}
     E(X_t) = \left\langle X_t \right\rangle = X_0  \mathrm{e}^{at},
\end{align}
where $X_0$ is the initial value at $t=0$.


\subsubsection*{Task a}

Solve the equation for $N_p$ realisations and a timestep $\Delta t$, and compare the mean of the distribution of solutions to the expectation value. Try this for different values of $N_p$, and different values of $\Delta t$. Do you observe any trends?

\subsubsection*{Task b}

If you want, you can try to compare the distribution of the numerical solutions to the expected distribution, which is log-normal (which means that $\log(X_t)$ is normally distributed). You can read about geometric Brownian motion, for example at Wikipedia, and try to compare against the distribution described there. Note that the Wikipedia page uses a little different notation.






\end{document}
