\documentclass[a4paper]{article}
\usepackage[margin=1.2in]{geometry}

\title{\vspace{-3em} Exercise 1\\[10pt] \large FY8503 Advanced theoretical physics\\  Transport modelling with Stochastic differential equations}

\author{September 11, 2023}
\date{}

\begin{document}
\maketitle

\section*{Problem 1}

There was a question at the first lecture about what happens if you construct a stochastic process with non-Gaussian increments. We will investigate this question in this exercise, by generating realisations of a random process, $X_t$, by adding increments from a non-Gaussian distribution with mean $\mu=0$ and variance $\sigma^2 < \infty$. Try the different tasks below with different distributions, for example:
\begin{itemize}
    \item Maxwell distribution
    \item Weibull distribution
    \item Laplace distribution
    \item Some discrete distribution (for example $X\in\{-a, 0, a\}$ with suitably selected probabilities)
    \item Some non-symmetric discrete distribution (for example $X\in\{-a, 0, 2a\}$ with suitably selected probabilities)
\end{itemize}

We will use the notation that $X_n$ is the numerical value of the process at time $t_n = n \Delta t$, and that the smallest increments (which we use to construct the numerical realisation of the process) are called $\Delta X_n = X_n - X_{n-1}$.

\subsubsection*{Task a}

Generate a single numerical approximation, $X_t$, of a stochastic process by starting at zero.
Construct the process from $t=0$ to $t=T$, by adding $N_t = T/\Delta t$ increments with $\mu=0$ and variance $\sigma^2 = \Delta t$. Plot the resulting process. Keep $T$ constant, but change $\Delta t$, and try observe how the process changes.

\subsubsection*{Task b}

Investigate the distribution of the increments. Clearly, the increments of two neighbouring points, \emph{e.g.}, $X_n - X_{n-1}$ are the same increments that the process was constructed from. But how about other increments, such as $X_n - X_{n-10}$ or $X_n - X_{n-100}$? Plot the distribution of the increments for a few different intervals. Use sufficiently many realisations that you get a nice and smooth histogram of the distribution. Is this distribution the same as the distribution you started out with for the smallest increments?

Investigate this using a few different distributions to construct the stochastic process. Note that you must choose the distribution such that it has $\mu=0$ and $\sigma^2=\Delta t$. This is possible even for non-symmetric discrete distributions with only two or three different outcomes, if you choose the values and the probabilities correctly.



\end{document}
