\documentclass[a4paper]{article}
\usepackage{amsmath}
\usepackage{url}
\usepackage{graphicx}
\usepackage[margin=1.2in]{geometry}

\title{\vspace{-3em} Exercise 5\\[10pt] \large FY8503 Advanced theoretical physics\\  Transport modelling with Stochastic differential equations}

\author{October 9, 2023}
\date{}

\begin{document}
\maketitle



\section*{Problem 1}

In this exercise, we will consider the concentration (density) of diffusing particles. The diffusion equation (for constant diffusivity) is given by

\begin{align}
    \label{eq:diff}
    \frac{\partial p}{\partial t} = K \frac{\partial^2 p}{\partial x^2}.
\end{align}


\subsubsection*{Task a}

Find the SDE corresponding to Eq.~\eqref{eq:diff}.

\subsubsection*{Task b}

Implement a solver for the SDE found above, with a reflecting boundary at $x=0$. Solve the SDE for a large number of particles, all starting out at some position $x_0 > 0$. Choose suitable values for $K$, $x_0$ and the integration time, such that you can clearly see the effect of the reflecting boundary.

Compare the density of particle positions to the analytical solution of Eq.~\eqref{eq:diff}, which in this case is a sum of two Gaussians, both with variance $\sigma^2=2Kt$, and with means $\mu_+ = x_0$ and $\mu_- = -x_0$. Note that using two Gaussians, symmetrically placed about the reflecting boundary in this way is the same idea as the use of ``image charges'' in electrostatics.

\subsubsection*{Task c}

Repeat the problem above, but now with an absorbing boundary at $x=0$. In this case, the analytical solution is a difference of two Gaussians, instead of the sum.




\end{document}
